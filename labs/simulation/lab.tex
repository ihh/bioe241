\chapter{Simulation}

Goal of this lab:
Simulate a discrete-state continuous-time Markov chain.

Consider the following exercises.
Provide pseudocode for at least two of them and implement at least one.

{\bf You will be asked to present your implemented simulation in the third week of class.}
Should you wish to present this in the form of an animation, several options are available:
you could generate series of images using e.g. a scripting language + library,
then use Berkeley MPEG encoder to stitch together into a movie;
or you could use one of various applications on desktop OS's that will do this.

\section{Simulate from an exponential distribution by inverting the cumulative distribution}

Pseudocode to be provided.

\section{Simulate the general reversible-time nucleotide model over a finite time interval}

\section{Simulate the general reversible-time nucleotide model over a phylogenetic tree}

\section{Simulate Gillespie's algorithm}

\cite{Gillespie77}

\section{Simulate the spatial Lotka-Volterra model on a 2D lattice}

Easy: discrete-time version.
Hard: continuous-time version.

Collect summary statistics.

\section{Simulate the 1D Ising model and the methylation-induced-CpG-deamination model}

It may be easiest to implement the general nearest-neighbor irreversible-time nucleotide model,
as both the Ising and CpG models are a subset of this.
