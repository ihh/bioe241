\chapter{Simulation}

Goal of this lab:
Simulate a discrete-state continuous-time Markov chain.

Consider the following exercises.
Provide pseudocode for at least two of them and implement at least one.

{\bf You will be asked to present your implemented simulation in the third week of class.}
Should you wish to present this in the form of an animation, several options are available:
you could generate series of images using e.g. a scripting language + library,
then use Berkeley MPEG encoder to stitch together into a movie;
or you could use one of various applications on desktop OS's that will do this.

\section{Simulate from an exponential distribution by inverting the cumulative distribution}

Sample a random variable $t$ representing the amount of time you would have to wait for a radioactive decay event
given that the number of such events in a time interval $[0,T]$ is Poisson-distributed with mean $rT$.

Hint 1: the probability density of $t$ is $p(t) = r \exp(-r t)$.

Hint 2: let $v$ be an r.v. uniformly distributed on $[0,1]$,
so that the pdf of $v$ is $q(v)$ which is 1 in the range $[0,1]$, and 0 outside.
We can easily sample from this distribution.
Now suppose that there is some function $f$ such that $v = f(t)$.
We want to choose $f$ such that $q(v)dv = p(t)dt$.
But we know $q(v)=1$ and $dv = f'(t) dt$.
Thus $f'=p$ and so $f(t) = \int_{-\infty}^t p(u) du$.
Inverting this function, i.e. finding $f^{-1}(v)$, gives us $t$ in terms of $v$.

Using these hints, or otherwise, show that $t=-\frac{\log(v)}{r}$ has the required density.
Write down pseudocode for generating sample values of $t$.

\section{Simulate a general nucleotide model over a finite time interval}

Suppose that $R$ is a $4 \times 4$ rate matrix
(with negatives on the diagonals, so $R_{ii} = -\sum_{j \neq i} R_{ij}$).

A process starts in an initial state $x(0)$ and is allowed to evolve for time $T$.
Write down pseudocode that generates a sample trajectory,
yielding not only the final value $x(T)$ but also all the timepoints at which the state changes.
It should also keep track of the number of times each substitution $i \to j$ occurred on the trajectory,
and the total amount of time that was spent in each state, ``waiting'' for the next mutation.

The rate matrix $R$ should be represented by a 2D array or other generic data structure in your code
(i.e. your code should allow for different rate matrices by just changing some variables;
however, you don't have to read it from a file).

If you'd like to start with a simpler model, use Kimura's two-parameter model \cite{Kimura80}.

\section{Simulate the general nucleotide model over a phylogenetic tree}

This exercise is a little harder than the previous one, though it is the same basic idea.
At each node in the (bifurcating) phylogeny, you will need to save the state, handle one branch,
restore the state \& handle the other branch.

For simplicity, you can hard-code the tree, e.g. you could use the tree {\tt ((A,B),(C,D))}
with appropriate branch lengths.

\section{Simulate Gillespie's algorithm}

Simulate Gillespie's algorithm \cite{Gillespie77}
for a simple set of chemical reactions, e.g.
\[
A + B \leftrightarrow AB 
\]
and
\[
AB + C \rightarrow ABC
\]
(with appropriate made-up rate constants).


\section{Simulate the methylation-induced-CpG-deamination model}

The goal here is to understand the temporal dynamics.
If the original sequence contains an $N$-mer $x_1 x_2 x_3 \ldots x_N$,
what is the probability that the descendant sequence has the $N$-mer $y_1 y_2 y_3 \ldots y_N$,
if the sequences are separated by evolutionary time $t$?

You will need to run lots of simulations to get good statistics on this.
(It may be easiest to implement the general nearest-neighbor irreversible-time nucleotide model,
as both the Ising and CpG models are a subset of this.)

You may use any model that allows for nearest-neighbor effects,
e.g. Kimura's two-parameter model \cite{Kimura80}
with an elevated rate for $CG \to CA$ and $CG \to TG$ mutations.
