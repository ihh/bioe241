\documentclass{beamer}

\input{../defs.tex}

% workaround for beamer bug
\providecommand\thispdfpagelabel[1]{}  % workaround

% presentation
\mode<presentation>
{
  \usetheme{Warsaw}
  % or ...

  \setbeamercovered{transparent}
  % or whatever (possibly just delete it)
}


\usepackage[english]{babel}
% or whatever

\usepackage[latin1]{inputenc}
% or whatever

\usepackage{times}
\usepackage[T1]{fontenc}
% Or whatever. Note that the encoding and the font should match. If T1
% does not look nice, try deleting the line with the fontenc.


\title[HMMs] % (optional, use only with long paper titles)
{Hidden Markov Models}

\subtitle
{Stochastic Regular Grammars} % (optional)

\author% [Holmes] (optional, use only with lots of authors)
{I.~Holmes} % \inst{1} \and S.~Another\inst{2}
% - Use the \inst{?} command only if the authors have different
%   affiliation.

\institute[University of California, Berkeley] % (optional, but mostly needed)
{
%  \inst{1}%
  Department of Bioengineering\\
  University of California, Berkeley}
% - Use the \inst command only if there are several affiliations.
% - Keep it simple, no one is interested in your street address.

\date%[Short Occasion] % (optional)
{Spring semester}

\subject{Talks}
% This is only inserted into the PDF information catalog. Can be left
% out. 



% If you have a file called "university-logo-filename.xxx", where xxx
% is a graphic format that can be processed by latex or pdflatex,
% resp., then you can add a logo as follows:

% \pgfdeclareimage[height=0.5cm]{university-logo}{university-logo-filename}
% \logo{\pgfuseimage{university-logo}}



% Delete this, if you do not want the table of contents to pop up at
% the beginning of each subsection:
\AtBeginSubsection[]
{
  \begin{frame}<beamer>{Outline}
    \tableofcontents[currentsection,currentsubsection]
  \end{frame}
}


% If you wish to uncover everything in a step-wise fashion, uncomment
% the following command: 

%\beamerdefaultoverlayspecification{<+->}


\begin{document}

\begin{frame}
  \titlepage
\end{frame}

\begin{frame}{Outline}
  \tableofcontents
  % You might wish to add the option [pausesections]
\end{frame}


\section{Single-sequence hidden Markov models}



\begin{frame}{}

\itemb
\item Early motivation: isochores
 \itemb
 \item Long regions of uniform GC content (which is correlated with gene density, recombination frequency...)
  \itemb
  \item e.g. Major Histocompatibility Complex (MHC) class II and class III sequences on human chromosome 6
  \item Lengths 900.9 kb, 642.1 kb; GC-content 41\%, 52\%
  \iteme
 \item Gary Churchill: first Hidden Markov Model for isochore detection (1989)
 \item Earliest non-thermodynamic hit to ``isochore'' on PubMed is 1986, Alonso {\em et al}
 \item HMM analogy: occasionally dishonest casino (Durbin {\em et al})
 \iteme
\iteme
\end{frame}

\begin{frame}{Notation}
\itemb
\item Hidden Markov model: notation
 \itemb
 \item Let $x$ denote hidden state, $y$ observed symbol. State space includes START and END
 \item Let $e(x,y)$ be probability of emitting character $y$ in state $x$
 \item Let $t(i,j)$ be probability of transition to state $j$ if currently in state $i$
 \iteme
\iteme
\end{frame}

\begin{frame}{}
\itemb
\item Idea of a particular partitioning as a ``path'' through the HMM
 \itemb
 \item Let $y_n$ be observed nucleotide at position $n$ and let $x_n$ be hidden state of position $n$
  \itemb
  \item Let $L$ be length of sequence
  \item For convenience, set $x_0=$START and $x_{L+1}=$END
  \item Let $Y=\{y_1 \ldots y_L\}$ represent entire observed sequence, $X=\{x_0 \ldots x_{L+1}\}$ hidden state sequence
  \item Each $X=\{x_k\}$ represents a ``path'' (sketch); there are $\sim K^L$ possible paths for $K$ states
  \iteme
 \item Joint likelihood is $P(X,Y) = t(x_0,x_1) \prod_{k=1}^L e(x_k,y_k) t(x_k,x_{k+1})$
  \inone{Note that $P(X,Y) \equiv P(X,Y|x_0)$. Conditioning on $x_0$ is assumed throughout}
 \iteme
\item Two questions, two algorithms. (i) Viterbi: what $X$ maximises $P(X,Y)$? (ii) Forward: what is $P(Y)$?
\iteme
\end{frame}

\begin{frame}{}
\tiny
Viterbi algorithm for finding ML hidden state path
\begin{eqnarray*}
\max_X P(X,Y) & = & \max_{x_L} \max_{x_{L-1}} \ldots \max_{x_1} t(x_0,x_1) \prod_{k=1}^L e(x_k,y_k) t(x_k,x_{k+1}) \\
& = & \max_{x_L} t(x_L,x_{L+1}) e(x_L,y_L) \max_{x_{L-1}} t(x_{L-1},x_L) e(x_{L-1},y_{L-1}) \ldots \max_{x_1} t(x_1,x_2) e(x_1,y_1) t(x_0,x_1) \\
& = & \max_{x_L} t(x_L,\mbox{END}) V_L(x_L)
\end{eqnarray*}
where
\[
V_n(x_n) = \left\{ \begin{array}{ll} \displaystyle
e(x_n,y_n) \max_{x_{n-1}} t(x_{n-1},x_n) V_{n-1}(x_{n-1}) & \mbox{if $n > 0$} \\
1 & \mbox{if $n=0$}
\end{array} \right.
\]
Note that
\[
V_n(x_n) = \max_{x_1 \ldots x_{n-1}} P(x_1 \ldots x_n, y_1 \ldots y_n | x_0)
\]
or, in words, $V_n(x)$ is the maximum likelihood of any path ending in state $x$ and emitting symbols $y_1 \ldots y_n$.

The ML state sequence, $\argmax_X P(X,Y)$, is recovered by {\bf traceback} from $V_L$ to $V_1$.
\end{frame}

\begin{frame}{}
\tiny
Forward algorithm for summing over all possible hidden state paths
\begin{eqnarray*}
P(Y) & = & \sum_X P(X,Y) \\
& = & \sum_{x_L} \sum_{x_{L-1}} \ldots \sum_{x_1} t(x_0,x_1) \prod_{k=1}^L e(x_k,y_k) t(x_k,x_{k+1}) \\
& = & \sum_{x_L} t(x_L,x_{L+1}) e(x_L,y_L) \sum_{x_{L-1}} t(x_{L-1},x_L) e(x_{L-1},y_{L-1}) \ldots \sum_{x_1} t(x_1,x_2) e(x_1,y_1) t(x_0,x_1) \\
& = & \sum_{x_L} t(x_L,\mbox{END}) F_L(x_L)
\end{eqnarray*}
where
\[
F_n(x_n) = \left\{ \begin{array}{ll} \displaystyle
e(x_n,y_n) \sum_{x_{n-1}} t(x_{n-1},x_n) F_{n-1}(x_{n-1}) & \mbox{if $n > 0$} \\
1 & \mbox{if $n=0$}
\end{array} \right.
\]
Note that
\[
F_n(x_n) = P(x_n, y_1 \ldots y_n | x_0) = \sum_{x_1 \ldots x_{n-1}} P(x_1 \ldots x_n, y_1 \ldots y_n | x_0)
\]
or, in words, $F_n(x)$ is the sum of likelihoods of all paths ending in state $x$ and emitting symbols $y_1 \ldots y_n$.

By analogy to Viterbi traceback, a ML state sequence can be sampled by
stochastic traceback from $F_L$ to $F_1$.
\end{frame}

\begin{frame}{}
Implementation issues
 \itemb
 \item probability underflow for long sequences
 \item slow floating-point multiply
 \item possible solutions:
  \itemb
  \item (possibly discretized) log-space scores
  \item bignum types
  \iteme
 \iteme
\end{frame}

\section{Posterior probabilities for single-sequence HMMs}

\begin{frame}{}

\small

Definition of the posterior probability that position $n$ is in state $k$: sum over paths
\[
P(x_n=k|Y) = \frac{\sum_X P(X,Y) \delta(x_n=k)}{P(Y)}
\]
Splitting the path into three parts: $<n$, $=n$ and $>n$
\begin{eqnarray*}
P(x_n|Y)
& = & \sum_{x_1 \ldots x_{n-1}} \sum_{x_{n+1} \ldots x_L} \\
& & \quad \frac{P(x_1 \ldots x_n,y_1 \ldots y_n|x_0)\ P(x_{n+1} \ldots x_{L+1},y_{n+1} \ldots y_L|x_n)}{P(Y)} \\
& = & \frac{F_n(x_n)\ B_n(x_n)}{P(Y)} \\
B_n(x_n) & = & P(x_{L+1},y_{n+1} \ldots y_L|x_n) \\
& = & \sum_{x_{n+1} \ldots x_L} P(x_{n+1} \ldots x_{L+1}, y_{n+1} \ldots y_L|x_n)
\end{eqnarray*}
\end{frame}

\begin{frame}{}
\itemb

\item Likewise,
\[
P(x_n,x_{n+1}|Y) = \frac{F_n(x_n)t(x_n,x_{n+1})e(x_{n+1},y_{n+1})B_{n+1}(x_{n+1})}{P(Y)}
\]
and (useful for compression)
\[
P(y_{n+1}=k|y_1 \ldots y_n) = \frac{\sum_i \sum_j F_n(i) t(i,j) e(j,k)}{\sum_i F_n(i)}
\]
\item The Backward algorithm
\[
B_n(x_n) = \left\{ \begin{array}{ll} \displaystyle
\sum_{x_{n+1}} t(x_n,x_{n+1}) e(x_{n+1},y_{n+1}) B_{n+1}(x_{n+1}) & \mbox{if $n < L$} \\
t(x_L,\mbox{END}) & \mbox{if $n=L$}
\end{array} \right.
\]
\iteme
\end{frame}

\begin{frame}{}
\tiny
The Baum-Welch training algorithm:
\[
P(X,Y) = \left( \prod_{i,j} t(i,j)^{u(i,j)} \right) \left( \prod_{i,k} e(i,k)^{f(i,k)} \right)
\]
where $u(i,j)$ is the number of transitions $i \to j$ and $f(i,k)$ is the number of emissions of character $k$ from state $i$.
\\
Sufficient statistics for EM algorithm are therefore
\begin{eqnarray*}
\hat{t}(i,j) & = & \sum_{n=0}^L P(x_n=i,x_{n+1}=j|Y) \\
\hat{e}(i,k) & = & \sum_{n=1}^L P(x_n=i|Y) \delta(y_n=k)
\end{eqnarray*}
where $\hat{t}$ and $\hat{e}$ are the posterior expectations of $u$ and $f$.
Dirichlet (mixture) priors can be used for $t$ and $e$.
% \item Lagrange multiplier derivation?
\\
Note that $\hat{t}(i,j) = \pderiv{\log P(X)}{\log t(i,j)}$ and $\hat{e}(i,k)=\pderiv{\log P(X)}{\log e(i,k)}$
\end{frame}

\begin{frame}{}
Implementation issues: log-space probability addition
\begin{eqnarray*}
\log(e^a+e^b) & = & \max(a,b) + \oplus(|a-b|) \\
\oplus(|a-b|) & = & \log(1+e^{-|a-b|})
\end{eqnarray*}
If scores are discretized, can implement $\oplus$ as a lookup table for speed.
\end{frame}

\begin{frame}{}
Null states
 \itemb
 \item Sparse transition matrices are good (Forward/Backward and Viterbi time complexities are $\propto$ no. of transitions)
 \item Convenience: reduce number of transitions (e.g. delete states of profile HMM)
 \item Need to do a {\em topological sort} of null states so that they're filled in the correct order
 \item Awkwardness: null cycles. These break the toposort
 \iteme
\end{frame}

\begin{frame}{}
\small
Can always eliminate null states as follows
\[
{\bf t} = {\bf a} + {\bf b} + {\bf c} + {\bf d}
\]
\[
{\bf a} = \left( \begin{array}{ll} \ast & 0 \\ 0 & 0 \end{array} \right),
{\bf b} = \left( \begin{array}{ll} 0 & \ast \\ 0 & 0 \end{array} \right),
{\bf c} = \left( \begin{array}{ll} 0 & 0 \\ \ast & 0 \end{array} \right),
{\bf d} = \left( \begin{array}{ll} 0 & 0 \\ 0 & \ast \end{array} \right)
\]
\[
{\bf t}' = \left( \begin{array}{ll} \ast & 0 \\ 0 & 0 \end{array} \right)
= {\bf a} + \sum_{n=0}^\infty {\bf bd}^n {\bf c} = {\bf a} + {\bf b} ({\bf I} - {\bf d})^{-1} {\bf c}
\]
where ${\bf a}$ (and ${\bf t}'$) contain emit$\to$emit transitions,
 ${\bf b}$ emit$\to$null,
 ${\bf c}$ null$\to$emit and
 ${\bf d}$ null$\to$null.
\end{frame}

\begin{frame}{}
\itemb
\item Factor graph representation of HMM; similarity to pruning/peeling and parsimony (Forward-Backward $\subset$ Sum-Product)
\item General-purpose HMM implementations: DART library (C++)
\iteme

\end{frame}

\section{Pair Hidden Markov models}

\begin{frame}{}

\itemb
\item Motivation: pairwise sequence alignment, pairwise genefinding, etc.
\item Let $x$ denote hidden state, $y$ character in sequence $Y$, $z$ character in sequence $Z$
\item Let $\Delta y(x)$ be 1 if state $x$ emits a character to $Y$, and 0 otherwise; likewise $\Delta z(x)=1$ iff $x$ emits to $Z$
\iteme
\end{frame}

\begin{frame}{}
\itemb
\item Emission probability $e(x,y,z)$ is defined as follows:
 \itemb
 \item If $\Delta y(x)=1$ and $\Delta z(x)=0$, then $x$ is called a {\bf delete} state and $e(x,y,z) \equiv e_d(x,y)$ is a function of $x$ and $y$ only
 \item If $\Delta y(x)=0$ and $\Delta z(x)=1$, then $x$ is called an {\bf insert} state and $e(x,y,z) \equiv e_i(x,z)$ is a function of $x$ and $z$ only
 \item If $\Delta y(x)=1$ and $\Delta z(x)=1$, then $x$ is called a {\bf match} state and $e(x,y,z) \equiv e_m(x,y,z)$ is a function of $x,y$ and $z$
 \item If $\Delta y(x)=0$ and $\Delta z(x)=0$, then $x$ is called a {\bf null} state and $e(x,y,z)$ is a function of $x$ only (typically just 1)
 \item We will assume for now that there are no null states
 (apart from START and END).
 \iteme
\iteme

\end{frame}

\begin{frame}{}

\itemb
\item As before, $t(i,j)$ is the probability of transition to state $j$ if currently in state $i$
\item Suppose sequence lengths are $K,L$ so observed data are $Y = \{ y_1 \ldots y_K \}$ and $Z = \{ z_1 \ldots z_L \}$
\item Again we have a state path $x_1, x_2 \ldots x_N$
 and for convenience we set $x_0=$START and $x_{N+1}=$END.
 \itemb
 \item Denote by $\Lambda_{kl}$ the event that there exists a {\em break} at $(k,l)$:
\[
\Lambda_{kl}\ \Rightarrow\ \exists n:
\ \sum_{i=1}^n \Delta y(x_i)=k,
\ \sum_{i=1}^n \Delta z(x_i)=l
\]
So $\Lambda_{kl}$ means that,
at some point $n$ on the state path, the model
has emitted $k$ symbols to $Y$ and $l$ symbols to $Z$.
 \iteme
\iteme

\end{frame}

\begin{frame}{}

\small
Viterbi
\[
V_{kl}(x_n) = \max_{x_1 \ldots x_{n-1}}
P \left( \Lambda_{kl},x_1 \ldots x_n,y_1 \ldots y_k,z_1 \ldots z_l | x_0 \right)
\]
Recursion (assuming no null states)
\[
V_{kl}(x_n) = \left\{ \begin{array}{ll} \displaystyle
e(x_n,y_k,z_l) \max_{x_{n-1}} t(x_{n-1},x_n) & \\ \quad \times
V_{k-\Delta y(x_n),l-\Delta z(x_n)}(x_{n-1}) & \mbox{if $k > 0$ or $l > 0$} \\
1 & \mbox{if $k=l=0$ and $x_n=$START} \\
0 & \mbox{if $k=l=0$ and $x_n\neq$START} \\
0 & \mbox{if $k < 0$ or $l < 0$}
\end{array} \right.
\]

\end{frame}

\begin{frame}{}

\small
Forward
\begin{eqnarray*}
F_{kl}(x_n)
& = & P \left( \Lambda_{kl},x_n,y_1 \ldots y_k,z_1 \ldots z_l | x_0 \right) \\
& = & \sum_{x_1 \ldots x_{n-1}}
P \left( \Lambda_{kl},x_1 \ldots x_n,x_{N+1},y_1 \ldots y_k,z_1 \ldots z_l | x_0 \right)
\end{eqnarray*}
Recursion (assuming no null states)
\[
F_{kl}(x_n) = \left\{ \begin{array}{ll} \displaystyle
e(x_n,y_k,z_l) \sum_{x_{n-1}} t(x_{n-1},x_n) & \\ \quad \times
F_{k-\Delta y(x_n),l-\Delta z(x_n)}(x_{n-1}) & \mbox{if $k > 0$ or $l > 0$} \\
1 & \mbox{if $k=l=0$ and $x_n=$START} \\
0 & \mbox{if $k=l=0$ and $x_n\neq$START} \\
0 & \mbox{if $k < 0$ or $l < 0$}
\end{array} \right.
\]

\end{frame}

\begin{frame}{}

\small
Backward
\begin{eqnarray*}
B_{kl}(x_n)
& = & P \left( \Lambda_{kl},x_{N+1},y_{k+1} \ldots y_K,z_{l+1} \ldots z_L | x_n \right) \\
& = & \sum_{x_{n+1} \ldots x_N}
P \left( \Lambda_{kl},x_{n+1} \ldots x_{N+1},y_{k+1} \ldots y_K,z_{l+1} \ldots z_L | x_n \right)
\end{eqnarray*}
Recursion (assuming no null states)
\[
B_{kl}(x_n) = \left\{ \begin{array}{ll} \displaystyle
\sum_{x_{n+1}}
t(x_n,x_{n+1})
e(x_{n+1},y_{k+1},z_{l+1}) & \\ \quad \times
B_{k+\Delta y(x_n+1),l+\Delta z(x_n+1)}(x_{n+1}) & \mbox{if $k < K$ or $l < L$} \\
t(x_n,\mbox{END}) & \mbox{if $k=K$ and $l=L$} \\
0 & \mbox{if $k > K$ or $l > L$}
\end{array} \right.
\]

\end{frame}

\begin{frame}{}

\tiny
Evidence, posterior probabilities \& EM counts
\begin{eqnarray*}
P(Y,Z) & = & \sum_x F_{KL}(x) t(x,\mbox{END}) \\
P(\Lambda_{kl},x_n|Y,Z) & = & \frac{F_{kl}(x_n) B_{kl}(x_n)}{P(Y)} \\
P(\Lambda_{kl},x_n,x_{n+1}|Y) & = & \frac{F_{kl}(x_n) t(x_n,x_{n+1}) e(x_{n+1},y_{k+1},z_{l+1})
B_{k+\Delta y(x_n+1),l+\Delta z(x_n+1)}(x_{n+1})}{P(Y)} \\
\hat{t}(i,j) & = & \sum_{k=0}^K \sum_{l=0}^L P(\Lambda_{kl},x_n=i,x_{n+1}=j|Y,Z) \\
\hat{e}_m(x,y,z) & = & \sum_{k:y_k=y}\ \sum_{l:z_l=z}\ P(\Lambda_{kl},x_n=x|Y,Z) \\
\hat{e}_d(x,y) & = & \sum_{k:y_k=y}\ \sum_{l=0}^L\ P(\Lambda_{kl},x_n=x|Y,Z) \\
\hat{e}_i(x,z) & = & \sum_{k=0}^K\ \sum_{l:z_l=z}\ P(\Lambda_{kl},x_n=x|Y,Z)
\end{eqnarray*}

\end{frame}

\begin{frame}{}

Decision theory (``optimal accuracy'').
 \itemb
 \item Decision theory: maximise expected ``reward'', making use of the posterior distribution
 \item Overlap score: an objective function (i.e. reward) that compares predicted alignment $\alpha$ with true alignment $\alpha'$
  \itemb
  \item Overlap score is $|\alpha \cap \alpha'|$, where an alignment is viewed as a set of match co-ords $\alpha = \left\{ (k_1,l_1), (k_2,l_2) \ldots \right\}$
  \item Several other good objective functions (e.g. ``Cline shift score''); overlap is simpler, albeit less realistic
   \inone{NB also $\delta(\alpha=\alpha')$ which only rewards perfect alignments, yielding a multiplicative, Viterbi-like recursion}
  \item Example criteria: how good is alignment for structure prediction? homology detection? benchmark of choice?
   \inone{e.g. PROBCONS (Batzoglou {\em et al}) uses the sum-of-pairs score, same as the BAliBASE benchmark}
  \iteme
\iteme
\end{frame}

\begin{frame}{}
\itemb
 \item Posterior expectation of overlap score for an alignment (NB only match states have $\Delta y(x) \Delta z(x) \neq 0$)
\begin{eqnarray*}
A[\alpha] & = & \sum_{(k,l) \in \alpha} P(\mbox{match},k,l) \\
P(\mbox{match},k,l) & = & \sum_x \Delta y(x) \Delta z(x) P(\Lambda_{kl},x_n=x)
\end{eqnarray*}
\iteme
\end{frame}

\begin{frame}{}
\itemb
 \item Optimal accuracy recursion: let $\alpha_{kl}$ be any alignment up to $(k,l)$, so $k' \leq k$ and $l' \leq l$ for all $(k',l') \in \alpha$. Then
\begin{eqnarray*}
O_{kl} & = & \max_{\alpha_{kl}} A\left[ \alpha_{kl} \right] \\
& = & \left\{ \begin{array}{ll}
\max \left( \begin{array}{l} O_{k-1,l-1} + P(\mbox{match},k,l), \\ O_{k,l-1}, \\ O_{k-1,l} \end{array} \right) & \mbox{if $k,l \geq 0$} \\
0 & \mbox{otherwise}
\end{array} \right.
\end{eqnarray*}
 \item Optimal alignment $\alpha$ recovered by traceback from $O_{KL}$.
 \iteme

\end{frame}

\begin{frame}{}

\itemb
\item Dynamic programming algorithms whose finite state automata are almost or exactly Pair HMMs
 \itemb
 \item Needleman-Wunsch; Smith-Waterman; Gotoh; Altschul, Proteins 1998
 \item General implementations: DART library (C++), Exonerate (C), HMMoC (Java/C++), ...
 \iteme
\iteme

\end{frame}

\section{Evolutionary Hidden Markov models}
\label{sec:EvolutionaryHMM}

\begin{frame}{}

\itemb
\item Can readily extend the Pair HMM to a multi-sequence HMM for multiple sequence alignment
 \itemb
 \item Arbitrary number $N$ of output sequences $Y^{(1)}, Y^{(2)}, Y^{(3)} \ldots Y^{(N)}$ of lengths $L_1 \ldots L_N$ (see e.g. Holmes 2003)
 \item Dynamic programming time/memory complexity is $O(\prod_n L_n)$---not cheap
 \item Ultimately, would like to structure $\Delta Y^{(n)}(x)$, $t(x,x')$ and $e(x,y^{(1)} \ldots y^{(N)})$ according to some underlying phylogenetic tree
  \inone{The DP algorithms can also be tree structured, c.f. ``progressive alignment''}
 \item For now, we ignore phylogenetic structure of indels ($\Delta$,$t$) and concentrate on substitution model ($e$)
 \iteme
\iteme
\end{frame}

\begin{frame}{}
\itemb
\item Initial, simplistic, restrictive concept of Evolutionary HMM, lacking a good gap model:
 \itemb
 \item {\bf A single-sequence HMM that emits phylogenetically-correlated multiple alignment columns, instead of single characters.}
 \item Follows e.g. Goldman, Thorne \& Jones, 1996 ({\em ``Using evolutionary trees in protein secondary structure prediction...''})
 \item For now, gaps will be glossed over heuristically (disallowed/treated as wildcards/excessively gappy columns discarded/etc.)
 \iteme
\item For a tree with $N$ leaves, the emitted symbol alphabet is $\Omega^N$ where $\Omega$ is the single-character alphabet
\iteme
\end{frame}

\begin{frame}
\itemb
\item Now the emission function $e(x,{\bf y)}$ for ${\bf y} \in \Omega^N$ is expensive (and unnecessary) to tabulate
 \itemb
 \item We can implement it as an instance of pruning instead: DP within DP
 \item Assume an underlying continuous-time discrete-state Markov chain with rate matrix ${\bf R}^{(x)}$ and initial distribution $\pi^{(x)}$
 \item Let $e(x,{\bf y})$ be probability of observing characters ${\bf y}$ at (leaf) nodes of a phylogenetic tree, with this process
  \inone{Tree, like alignment, will be specified as an input to our DP recursions}
 \iteme
\item The counts $\hat{t}(x,x')$ are still relevant, but $\hat{e}(x,{\bf y})$ are used to accumulate EM update counts for ${\bf R}^{(x)}$
 \inone{Can update $t$'s and ${\bf R}^{(x)}$'s simultaneously or asynchronously; c.f. Neal and Hinton}
\iteme
\end{frame}

\begin{frame}{}
\itemb
\item Many applications in genomic biology
 \itemb
 \item Annotation of multiple alignments of DNA, RNA or protein sequences
 \item Isochores; gene prediction; DNA-protein binding site modeling and analysis; protein transmembrane structure, signal peptide, domain profiles... etc.
 \iteme
\iteme
\end{frame}{}

\begin{frame}
\itemb
\item Implementation: {\tt xrate} (distributed with the DART library)
 \itemb
 \item S-expressions for the underlying alignment grammar
  \itemb
  \item Here ``alignment grammar'' means the alphabet $\Omega$, the HMM transition matrix $t$, the $\Delta$'s, the ${\bf R}$'s and the $\pi$'s
  \item More generally, can use stochastic context-free grammars as well as HMMs, \& states can emit several co-evolving columns
  \iteme
 \item Stockholm format for alignment, tree, Viterbi annotation, likelihoods and posterior probabilities
  \inone{Allows internal nodes as well as leaves to be specified}
 \item By default, gaps in the multiple alignment are treated as wildcards (unobserved character is summed over)
  \inone{A smarter handling of gaps soon leads us to indel rate models and so-called {\bf ``Statistical Alignment''}}
 \item Log messages (type {\tt xrate -help} and {\tt xrate -loghelp})
 \item Command-line options (type {\tt xrate -help})
 \iteme
\iteme

\end{frame}

\section{Discriminative models and conditional random fields}
\label{sec:DiscriminativeHMMs}

\begin{frame}{}

\itemb
\item HMMs model $P(Y) = \sum_X P(X,Y)$ ({\em generative} modeling). ML training maximizes this probability.
\item Intuitively, since we are interested in predicting $X$ correctly, it may make more sense to model conditional probability $P(X|Y)$ ({\em discriminative} modeling)
\item Consider the conditional likelihood for an HMM, expressed in terms of the {\em feature vector} $\{u,f\}$ implied by $X$:
\begin{eqnarray*}
\log P(X|Y) & = & \frac{1}{P(Y)} \exp \left( \sum_{i,j} u(i,j) \log t(i,j) + \right. \\ & & \left. \sum_{i,k} f(i,k) \log e(i,k) \right)
\end{eqnarray*}
where $P(Y)$ is computed by the Forward algorithm.
\iteme
\end{frame}

\begin{frame}{}
\itemb
\item We can write down a likelihood $P(X|Y)$ for a similarly trellis-structured graphical model as follows
\[
P(X|Y) = \frac{1}{Z} \exp \left( \sum_{i,j} u(i,j) a(i,j) + \sum_{i,k} f(i,k) b(i,k) \right)
\]
where $Z$ is a partition function (computed by a Forward-like sum-product algorithm)
\[
Z = \sum_{X'} \exp \left( \sum_{i,j} u_{X'}(i,j) a(i,j) + \sum_{i,k} f_{X'}(i,k) b(i,k) \right)
\]
For equivalence with the HMM, set $a(i,j)=\log t(i,j)$ and $b(i,k)=\log e(i,k)$.
\iteme
\end{frame}{}

\begin{frame}
\itemb
\item A {\em linear-chain conditional random field} is essentially such a trellis-structured model,
lacking the normalization constraints on the parameters $\theta=\{a,b\}$ that are implicit in the generative HMM
 \itemb
 \item Sutton \& McCallum, ``An Introduction to Conditional Random Fields for Relational Learning''
 \item Lafferty, McCallum \& Pereira, ``Conditional Random Fields: Probabilistic Models for Segmenting and Labeling Sequence Data''
 \iteme
\item To avoid overfitting, and in place of the normalization constraints,
it is useful to add a {\em regularization} term, e.g. a Gaussian prior on $a()$ and $b()$ with variance $\sigma^2$, penalizing large weights.
Then the function to be optimized is
\[
\ell(\theta) = \log P(X|Y) - \sum_{i,j} \frac{a(i,j)^2}{2\sigma^2} + \sum_{i,k} \frac{b(i,k)^2}{2\sigma^2}
\]
\iteme
\end{frame}

\begin{frame}
\itemb
\item Parameter estimation proceeds by {\bf numerical optimization} of $\ell(\theta)$, typically using quasi-Newton algorithms
 \inone{e.g. the BFGS algorithm: Dimitri P. Bertsekas. Nonlinear Programming. Athena Scientific, 2nd edition, 1999}
\item The partial derivatives $\pderiv{\ell}{a(i,j)}$ and $\pderiv{\ell}{b(i,k)}$ are computed by direct analogy to $\hat{t}(i,j)$ and $\hat{u}(i,k)$ in Baum-Welch.
\iteme
\end{frame}{}

\begin{frame}
\itemb
\item In the absence of normalization constraints, we are free to add more ``features'' without having to directly account for them by subdividing the state space of the HMM.
For example: a run of T's, a palindromic motif, etc.
In the generative (HMM) view, all such features must be explicitly identified with a path through the model, so that nothing is counted more than once.
In a discriminative framework, it doesn't matter if a residue is counted twice.
\item In the trellis factor graph view, functions relating $x_i$ and $y_i$ are $P(x_i|Y)$ rather than $P(y_i|x_i)$
\item HMMs and linear CRFs form what Ng and Jordan (2002) call a ``generative-discriminative pair''.
 \inone{Other such pairs include naive Bayes {\em vs} logistic regression.}
\iteme

\end{frame}

\section*{Summary}

\begin{frame}{Summary}

  % Keep the summary *very short*.
  \begin{itemize}
  \item HMMs
  \end{itemize}

\end{frame}


\end{document}
